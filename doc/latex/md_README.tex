P\+R\+O\+J\+ET \+: Mini\+S\+Hell « my\+\_\+sh » – à réaliser en binôme – Durée approximative $\sim$ 8h00min.

A l’aide des supports de cours et des mémentos et des exercices achevés en TP et du manuel Linux réalisez \+: (Tout code ou implémentation compilant ou non sera étudié, toute fois le barème sera adapté) Le présent sujet de projet comporte 3 pages

\subsection*{I – Synopsis}

L’objectif de ce projet et de réaliser -\/dans une proportion simplifiée-\/ l’implémentation d’un interpréteur de commande similaire à bash. Au lancement ce dernier doit pouvoir afficher un prompt attendant la saisie d’une commande ou d’un sous-\/ensemble de commandes.

Exemple \+: Prompt$>$ ls –a ; who Prompt$>$ ls –a $\vert$ grep toto Prompt$>$ date

\subsection*{II – Démarche}

Les commandes sont lancées par des processus fils laissant au père le rôle d’interpréteur. Ce dernier doit attendre la fin du/des processus fils pour afficher le résultat d’exécution du/des commande(s) soumises. Certaines commandes et variables devront être internes (fonctionnalités built-\/in), c’est-\/à-\/dire, directement prises en compte par le code. Ainsi on distinguera trois phases distinctes \+: \begin{DoxyVerb}L’évaluation de l’expression soumise à l’interpréteur.
L’exécution ordonnancée du sous-ensemble de commandes.
La soumission du résultat d’exécution.
\end{DoxyVerb}


\subsection*{I\+II – Résultats attendus}

Le livrable attendu pour ce projet se résume en un code source compilable et exécutable répondant d’une part aux fonctions métiers suivantes \+:


\begin{DoxyItemize}
\item F\+M01 – Le binaire est capable d’exécuter une commande simple (ie \+: ls –l ; ps ; who) ✓
\item F\+M02 – Le binaire est capable d’exécuter un sous-\/ensemble de plusieurs commandes de sorte à prendre en compte \+: ✓
\begin{DoxyItemize}
\item Les opérateurs de contrôle \+: \&\& et $\vert$$\vert$
\item Les redirections de flux simples \+: $\vert$, $>$, $<$, $>$$>$, $<$$<$
\item L’exécution en arrière-\/plan \+: \&
\end{DoxyItemize}
\item F\+M03 – L’exécution des commandes internes (fonctionnalités built-\/in) suivantes \+: ✓
\begin{DoxyItemize}
\item cd -\/ Permettant de se déplacer au sein d’une arborescence de fichier.
\item pwd – Affichant la valeur de la variable contenant le chemin du répertoire courant.
\item exit – Permettant de quitter l’interpréteur.
\item echo – Permettant d’afficher du texte sur la sortie standard.
\end{DoxyItemize}
\item F\+M04 -\/ La persistance des commandes saisie dans un fichier (historique) ✓
\end{DoxyItemize}

D’autres fonctionnalités optionnelles peuvent êtres implémentés \+:


\begin{DoxyItemize}
\item F\+O01 – La réalisation d’un mode batch (ie \+: ./my\+\_\+shell –c « ls –al $\vert$ grep toto ») ✓
\item F\+O02 – La création de variables d’environnement
\item F\+O03 – La prise en charge d’alias
\end{DoxyItemize}

Concernant les exigences techniques attendues, vous devez respecter les contraintes suivantes \+:


\begin{DoxyItemize}
\item C\+T01 – La compilation du projet doit se faire via un Makefile. ✓
\item C\+T02 – La définitions des structures doit se faire dans un fichier \hyperlink{typedef_8h}{typedef.\+h}. ✓
\item C\+T03 – La définition des méthodes protoype (.h) \& implémentation (.c) doit se faire de manière séparée autant que faire se peut. ✓
\item C\+T04 – Le code produit doit être documenté. $\sim$
\item C\+T05 – La gestion des erreurs doit se faire via « les mécanismes proposés par errno ». ✓
\end{DoxyItemize}

D’autres contraintes techniques peuvent être prises en compte \+:
\begin{DoxyItemize}
\item C\+T\+O01 – La documentation du code générée via l’utilitaire doxygen.
\item C\+T\+O02 – Le code est soumis à un contrôle de couverture via l’utilitaire gcov.
\item C\+T\+O03 – Une page de manuel Linux est rédigée pour détailler l’exécution du shell.
\end{DoxyItemize}

\subsection*{IV – Evaluation}

Ce projet est à réaliser en binôme et donnera lieu à une présentation d’environs 15min répartie en deux phases \+: présentation\&démonstration suivi d’un partie consacrée aux questions. (Un support de présentation pourra être utilisé mais ne devra pas contenir plus de 5 diapos).

La réalisation de ce shell simplifié vise à mettre en œuvre l’ensemble des connaissances abordées au cours de ce module. La restitution du plus grand nombre de notions à travers le code produit vous permet de valider vos compétences.

Ainsi la réalisation de l’ensemble des fonctionnalités métiers FM 1 à 4 ainsi que le respect des contraintes techniques vous assure une note supérieure à la moyenne signifiant l’acquis des connaissances. Toutefois la réalisation des fonctionnalités métiers et/ou optionnelles vous permettent d’augmenter votre note le cas échéant. 